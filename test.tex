\chapter{Introduction}
    \section{Introduction to Problem Domain}
	\subsection{Face Detection}
    \subsection{Open-Set Facial Recognition}
Facial recognition is a field that has enjoyed rapid development in the recent past, with the state-of-the-art improving rapidly in the last few years.  Most studies have focused on closed-set problems or face verification, while there have been few who have focused on the open-set problem.
\\
\\
Closed-set can be defined as circumstances where each sample face corresponds to a known identity.  Open-set recognition refers to circumstances where the sample face does not necessarily correspond to a known identity.  This means that the option must exist to reject a sample as not being recognized.  This is much closer to real-life circumstances.  A typical use-case could be to raise an alarm when one of a a set of specific subjects is identified, while ignoring all other negative samples.
\\
\\
Open-set facial recognition provides an extra layer of difficulty in achieving accurate results.
	\subsection{Constraints}
    \section{Problem Statement}
    \subsection{Introduction}
    \subsection{"Desired Results (Hypothesis?)"}
    \subsection{Proposed Architecture}
%__________________________________________________%
%________________ Research Methodology_____________%
    \chapter{Research Methodology}
    \subsection{Introduction}
Following a research methodology will ensure that the study is completed in a timely fashion.  The chosen methodology for this project is the Design Science Research method, as laid out by Hevner et al
%__________________________________________________%
%________________ Literature Study _____________%
    \chapter{Literature Study}
In this chapter, the topics introduced in \textit{Chapter 1} are researched in depth. 
\\The aim of the chapter is to gain a better understanding into each topic which is relevant to the problem domain
    \section{Existing Solutions}
    \section{Facial Recognition Engines}
    \subsection{Proprietary}
    \subsection{Open Source}
    \subsection{Selection of Engine}
    \section{Constraints}
    \subsection{Cost constraints}
    \subsection{Uncontrolled Environment}
    \subsection{Demographics}
    \section{Lack of Best Practice Architecture}
    \section{Allocation of Computational Resources}
%__________________________________________________%
%________________ Design _____________%
    \chapter{Design}
%__________________________________________________%
%________________ Evaluation and Results _____________%
    \chapter{Evaluation and Results}
    This purpose of this chapter is to document the performance of the completed system.  In this chapter, the results of 
%__________________________________________________%
%________________ Conclusion _____________%
    \chapter{Conclusion}
%__________________________________________________%